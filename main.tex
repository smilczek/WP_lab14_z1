\documentclass[12p,a4paper]{article}

% Language setting
% Replace `english' with e.g. `spanish' to change the document language
\usepackage[MeX]{polski}

% Set page size and margins
\usepackage[a4paper,top=2cm,bottom=2cm,left=3cm,right=3cm,marginparwidth=1.75cm]{geometry}

% Useful packages
\usepackage{indentfirst}
\usepackage{amsmath}
\usepackage{graphicx}
\usepackage[colorlinks=true, allcolors=blue]{hyperref}

\title{Podstawowa analiza matematyczna}
\author{Szymon Milczek}

\begin{document}
\maketitle

\newpage

\begin{abstract}
\large
Niniejszy artykuł nie zawiera rozdziałów, ponieważ klasa article nie posiada zdefiniowanej komendy article, zamiast artykułów mamy sekcje, podzielone na podsekcje, podzielone na podpodsekcje.

Podstawą analizy matematycznej (zdaniem autora) są \textbf{pochodne} oraz \textbf{całki}. Jeżeli nabędzie się odpowiednie zrozumienie tych operacji (które konceptualnie są do siebie \textit{bardzo} podobne, lecz w obliczeniach całki są \textit{o wiele} trudniejsze) znacznie ułatwi naukę wielu bardziej skomplikowanych dziedzin matematyki. Albo wyrażając się nieco \textit{brutalniej}, brak zrozumienia czym te operacje są na dobrą sprawę uniemożliwi głębszą edukację, nie tylko analizy matematycznej, ale również fizyki.

\end{abstract}

\newpage

\tableofcontents

\newpage

\section{Pochodne}

\subsection{Czym właściwie jest pochodna?}

\subsubsection{Problem ze szkolnym wyjaśnieniem \cite{Milczek22}}

W liceum wielu nauczycieli pokaże uczniom następujący wzór
$$f'(x) = \lim_{x_2 \to x_1} \frac{f(x_2) - f(x_1)}{x_2 - x_1},$$
określają go mianem \textit{"definicji pochodnej"} i każą rozwiązywać zadania. Uczeń wtedy pozostaje jedynie z myślami\\
\\
\textit{- no dobra, ale po co ja to właściwie robię?}\\
\\
Które przez moment jeszcze gdzieś obijają się w głowie, zanim uczeń po prostu zaakceptuje ten stan rzeczy i dalej będzie rozwiązywał zadania już bez zmrużenia okiem, bo tego wymaga od niego edukacja.

A życie przecież byłoby znacznie ułatwione, gdyby tylko ktoś wyjaśnił co właściwie taka pochodna mówi nam o równaniu. Wtedy stawałoby się o wiele jaśniejsze co się dzieje na lekcji i do czego taka pochodna może zostać użyta.

\subsubsection{Zrozumieć czym jest pochodna}
Aby zrozumieć czym jest pochodna trzeba przyjrzeć się definicji, i postarać się ją \underline{wyjaśnić}. Na razie zignorujmy limes. To co robimy, wybieramy 2 wartości, $x_1$ i $x_2$, dobieramy wartości jakie funkcja przyjmuje dla tych $x$, a następnie patrzymy jak zmieniła się wartość funkcji w zależności od zmiany $x$. Mówiąc bardzo luźno, badamy śrędnią zmianę wartości funkcji na odcinku pomiędzy $x_1$ a $x_2$. Im bliżej siebie wybierzemy $x$, tym odcinek na którym badamy zmianę funkcji będzie naturalnie \textit{mniejszy}. Jeżeli obydwa $x$ będą bardzo blisko siebie, chciałoby sie rzec \textit{nieskończenie} blisko, to zbadana zostanie średnia zmiana funkcji na odcinku $x_1 \to x_2$, który będzie tak mały, że można go określić punktem.

\noindent \textbf{Pochodna więc mówi nam z jaką prędkością zmienia się funkcja w danym punkcie.}

\subsection{Lepszy zapis}
W szkołach uczą zazwyczaj tylko zapisu pochodnej jako $f'(x)$. Wszystko z tym zapisem wydaje się być w porządku, dopóki ktoś nie zapyta:\\
\textit{Ale właściwie to po jakiej zmiennej pochodą liczymy?}\\
Istotnie, nie wiemy po jakiej! Tutaj akurat łatwo się domyślić, ponieważ funckja przyjmuje tylko jedną zmienną, ale co jeśli przyjmowałaby ich dwie?

Dlatego uważam że lepszym zapisem jest $\frac{d}{dx}f$, co czasem można też zapisać jako $\frac{df}{dx}$. Nietylko od razu widać po jakiej zmiennej liczona jest pochodne, ale również można zauważyć że sam wzór przedstawia czym jest sam w sobie!

O czym mówię? Proszę się przyjrzeć, w tym zapisie $d$ oznacza że brana jest jakaś mała wartość, $d$ jak $\Delta$, która zawsze jest stosowana dla określenia bardzo małej zmiany wartości, jak np we wzorze $f'(x) = \lim_{\Delta x \to 0} \frac{f(x + \Delta x) - f(x)}{\Delta x}$, który jeśli pod $\frac{f(x + \Delta x) - f(x)}{\Delta x}$ podstawimy $\Delta f$, bo przecież tak oznaczamy małe zmiany wartości, przyjmie postać $f'(x) = \lim_{\Delta x \to 0 \Delta f \to 0} \frac{\Delta f}{\Delta x}$
Wygląda znajomo? Oczywiście, jest to przecież nasz zapis $\frac{df}{dx}$! Oznacza on \textit{bardzo} małą zmianę wartości funkcji $f(x)$ podzieloną przez \textit{bardzo} małą zmianę wartości $x$.

\begin{table}[h]
\centering
\begin{tabular}{c|c}
Zapis & Wycena \\\hline
$\frac{df}{dx}$ & Dobry \\
$\frac{d}{dx}f$ & Lepszy \\
$f'(x)$ & Zły
\end{tabular}
\caption{\label{tab:zapis}Poszczególne sposoby zapisu z oceną od autora}
\end{table}

\section{Spójny tekst}

\subsection{Za mało punktów za taką robotę \cite{Szymon22}}

Ponieważ za 5 stron spójnego tekstu jest przyznawana tylko $\frac{1}{10}$ wszystkich punktów za zadanie, zajmę się teraz spełnianiem pozostałych podpunktów. Przykro mi, ale po prostu niewarto, tak bardzo jak chciałbym kontynuować swój wywód o analizie matematycznej, potrzebuję snu, a mam tylko 2 dni.

\subsection{Załączanie zdjęcia}

Aby załączyć zdjęcie, należy najpierw posiadać plik ze zdjęciem na dysku, a następnie użyć odpowiedniej komendy.

\begin{figure}
\centering
\includegraphics[width=0.5\textwidth]{frog.jpg}
\caption{\label{fig:frog}Zdjęcie żabki}
\end{figure}

\subsection{Obrazek wkomponowany w tekst}

Nie do końca rozumiem o co chodzi, ale chyba mam po prostu dać załączyć obrazek, który nie będzie się wyświetlał na samej górze strony, tylko pomiędzy blokami tekstu?

\begin{figure}[h!]
\centering
\includegraphics[width=0.7\textwidth]{szkic.png}
\caption{\label{fig:szkic}Autorski szkic kamieni}
\end{figure}
Także to by chyba było tyle?

\section{Podsumowanie}
Oto moje podsumowanie. Parę referencyj do obrazków i do tabelki:\\
Żabka\ref{fig:frog}\\
Szkic\ref{fig:szkic}\\
Tabelka\ref{tab:zapis}\\


\bibliographystyle{alpha}
\bibliography{sample}

\end{document}